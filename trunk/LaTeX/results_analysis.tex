\subsection{Results analysis}

Comparison plots have been slightly filtered using weak Savitzky-Golay filtering to reduce white noise for easier comparison. The line map and bar plots are without this filtering.

\subsubsection{Comparing different locations on sample R6-Q3-210}


\begin{figure}[H]
\centering
\includegraphics[width=0.7\columnwidth]{R6_comparisons}
\caption[R6-Q3-210 comparisons]{Comparison of different locations on sample R6-Q3-210 A from results in figure \ref{fig:R6-Area3_clean_area-20s}, \ref{fig:R6-Area1_dislocation_line-20s}, and \ref{fig:R6-Area2_dislocation_dot-20s}.}
\label{fig:R6_comparisons}%
\end{figure}

\begin{figure}[H]
\centering
\includegraphics[width=0.7\columnwidth]{R6_comparisons_Darea}
\caption[R6-Q3-210 comparisons close]{A closer look on differences in R6-Q3-210 A from graph in figure \ref{fig:R6_comparisons} }
\label{fig:R6_comparisons_Darea}%
\end{figure}

\begin{figure}[H]
\centering
\includegraphics[width=0.7\columnwidth]{R6_comparisons_TO}
\caption[R6-Q3-210 comparisons close]{A closer look on differences in R6-Q3-210 A from graph in figure \ref{fig:R6_comparisons} }
\label{fig:R6_comparisons_TO}%
\end{figure}

\begin{figure}[H]
\centering
\includegraphics[width=0.7\columnwidth]{R6_comparisons_I0}
\caption[R6-Q3-210 comparisons close]{A closer look on differences in R6-Q3-210 A from graph in figure \ref{fig:R6_comparisons} }
\label{fig:R6_comparisons_I0}%
\end{figure}


\subsubsection{Comparing different locations on sample ES1-Q3-201}

\begin{figure}[H]
\centering
\includegraphics[width=0.7\columnwidth]{ES1_comparisons}
\caption[ES1-Q3-201 comparisons]{Comparison of different locations on sample ES1-Q3-201 C from results in figure \ref{fig:ES1-Area1_dislocation_free-10s}, \ref{fig:ES1-Area4_dislocation_line-10s}, \ref{fig:ES1-Area2_dislocation_spot-10s}, and \ref{fig:ES1-Area3_grain_boundary-10s}. }
\label{fig:ES1_comparisons}%
\end{figure}

\begin{figure}[H]
\centering
\includegraphics[width=0.7\columnwidth]{ES1_comparisons_Darea}
\caption[ES1-Q3-201 comparisons close]{A closer look on differences in ES1-Q3-201 C from graph in figure \ref{fig:ES1_comparisons} }
\label{fig:ES1_comparisons_Darea}%
\end{figure}

\begin{figure}[H]
\centering
\includegraphics[width=0.7\columnwidth]{ES1_comparisons_TO}
\caption[ES1-Q3-201 comparisons close]{A closer look on differences in ES1-Q3-201 C from graph in figure \ref{fig:ES1_comparisons} }
\label{fig:ES1_comparisons_TO}%
\end{figure}

\begin{figure}[H]
\centering
\includegraphics[width=0.7\columnwidth]{ES1_comparisons_I0}
\caption[ES1-Q3-201 comparisons close]{A closer look on differences in ES1-Q3-201 C from graph in figure \ref{fig:ES1_comparisons} }
\label{fig:ES1_comparisons_I0}%
\end{figure}


\subsubsection{Comparing different locations on sample MH2-Q3-210}

\begin{figure}[H]
\centering
\includegraphics[width=0.7\columnwidth]{MH2_comparisons}
\caption[MH2-Q3-210 comparisons]{Comparison of different locations on sample MH2-Q3-210 B2 from results in figure \ref{fig:MH2-Area1-dislocation_free-20s}, \ref{fig:MH2-Area2-dislocation_line-20s}, and \ref{fig:MH2-Area3-dislocation_dot-20s} }
\label{fig:MH2_comparisons}%
\end{figure}


\begin{figure}[H]
\centering
\includegraphics[width=0.7\columnwidth]{MH2_comparisons_Darea}
\caption[MH2-Q3-210 comparisons close]{A closer look on differences in MH2-Q3-210 B2 from graph in figure \ref{fig:MH2_comparisons} }
\label{fig:MH2_comparisons_Darea}%
\end{figure}

\begin{figure}[H]
\centering
\includegraphics[width=0.7\columnwidth]{MH2_comparisons_TO}
\caption[MH2-Q3-210 comparisons close]{A closer look on differences in MH2-Q3-210 B2 from graph in figure \ref{fig:MH2_comparisons} } 
\label{fig:MH2_comparisons_TO}%
\end{figure}

\begin{figure}[H]
\centering
\includegraphics[width=0.7\columnwidth]{MH2_comparisons_I0}
\caption[MH2-Q3-210 comparisons close]{A closer look on differences in MH2-Q3-210 B2 from graph in figure \ref{fig:MH2_comparisons} } 
\label{fig:MH2_comparisons_I0}%
\end{figure}



%% Line plots

\begin{figure}[H]
\centering
\includegraphics[width=0.7\columnwidth]{MH2-mapping-bars-small_steps-20s}
\caption[MH2-Q3-210 line mapping]{Sample MH2-Q3-210 B2 pumped with 170mW at 14K line map using 10 small steps, looking at TO and BE line only from results in figure \ref{fig:MH2-mapping-small_steps-20s}.}
\label{fig:MH2-mapping-bars-small_steps-20s}%
\end{figure}

\begin{figure}[H]
\centering
\includegraphics[width=0.7\columnwidth]{MH2-mapping-bars-5_small_steps-20s}
\caption[MH2-Q3-210 line mapping]{Sample MH2-Q3-210 B2 pumped with 170mW at 14K line map using 20 small steps exactly 5 times larger than in figure \ref{fig:MH2-mapping-small_steps-20s}, looking at TO and BE line only from results in figure \ref{fig:MH2-mapping-5_small_steps-20s}.}
\label{fig:MH2-mapping-bars-5_small_steps-20s}%
\end{figure}


\begin{figure}[H]
\centering
\includegraphics[width=0.7\columnwidth]{MH2-mapping-bars-TOonBE-5_small_steps-20s}
\caption[MH2-Q3-210 line mapping]{Sample MH2-Q3-210 B2 comparing relative intensity between TO and BE line from results in \ref{fig:MH2-mapping-bars-5_small_steps-20s}.}
\label{fig:MH2-mapping-bars-TOonBE-5_small_steps-20s}%
\end{figure}


\subsubsection{Comparing similar areas on different samples}


\begin{figure}[H]
\centering
\includegraphics[width=0.7\columnwidth]{clean_area-20s}
\caption[Comparisons in a clean area]{Results from figure ( \ref{fig:R6-Area3_clean_area-20s},\ref{fig:ES1-Area1_dislocation_free-10s},\ref{fig:MH2-Area1-dislocation_free-20s}) where results in figure \ref{fig:ES1-Area1_dislocation_free-10s} are multiplied by 2, to account for 10s integration time compared to 20.}
\label{fig:clean_area-20s_comparison}%
\end{figure}

\begin{figure}[H]
\centering
\includegraphics[width=0.7\columnwidth]{clean_area-20s_zoom}
\caption[Comparisons in a clean area]{Plot from figure \ref{fig:clean_area-20s_comparison} in greater detail}
\label{fig:clean_area-20s_zoom_comparison}%
\end{figure}

\begin{figure}[H]
\centering
\includegraphics[width=0.7\columnwidth]{dislocation_dot-20s}
\caption[Comparisons in a dislocation dot]{Results from figure ( \ref{fig:R6-Area2_dislocation_dot-20s},\ref{fig:ES1-Area2_dislocation_spot-10s},\ref{fig:MH2-Area3-dislocation_dot-20s}) where results in figure \ref{fig:ES1-Area2_dislocation_spot-10s} are multiplied by 2, to account for 10s integration time compared to 20. }
\label{fig:dislocation_dot-20s_comparison}%
\end{figure}

\begin{figure}[H]
\centering
\includegraphics[width=0.7\columnwidth]{dislocation_dot-20s_zoom}
\caption[Comparisons in a dislocation dot]{Plot from figure \ref{fig:dislocation_dot-20s_comparison} in greater detail}
\label{fig:dislocation_dot-20s_zoom_comparison}%
\end{figure}

\begin{figure}[H]
\centering
\includegraphics[width=0.7\columnwidth]{dislocation_line-20s}
\caption[Comparisons in a dislocation line]{Results from figure ( \ref{fig:R6-Area1_dislocation_line-20s},\ref{fig:ES1-Area4_dislocation_line-10s},\ref{fig:MH2-Area2-dislocation_line-20s}) where results in figure \ref{fig:ES1-Area4_dislocation_line-10s} are multiplied by 2, to account for 10s integration time compared to 20. }
\label{fig:dislocation_line-20s_comparison}%
\end{figure}


\begin{figure}[H]
\centering
\includegraphics[width=0.7\columnwidth]{dislocation_line-20s_zoom}
\caption[Comparisons in a dislocation line]{Plot from figure \ref{fig:dislocation_line-20s_comparison} in greater detail}
\label{fig:dislocation_line-20s_zoom_comparison}%
\end{figure}

\begin{figure}[H]
\centering
\includegraphics[width=\columnwidth]{TO_comparisons}
\caption[Comparison or relative strength]{Comparison of relative strength of known characteristics to TO line}
\label{fig:TO_comparisons}%
\end{figure}