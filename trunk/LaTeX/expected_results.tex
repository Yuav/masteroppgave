\subsection{Expected results}


\subsubsection{Sample from clean feedstock}

Having carbon values around 2.26\cdot10$^{17}$~cm$^{-3}$, it is possible that the two-carbon atom band is visible here also. Else this sample is expected to only show intrinsic values similar to those showed by \cite{dean67} in so called 'good' areas due to low concentration of impurities. However, there might be precipitates and higher concentration of impurities at the grain boundaries and dislocations. Particularly heavy metals like Fe and Al can be detected here. It is expected that the band to band recombination from silicon show lower luminescence intensity for these areas.




\subsubsection{Phosphorus and boron doped samples}

With a high concentrations of doping atoms, it is expected that lines attributed to phosphorous and boron atoms in the photoluminescence spectra is detectable. A previous study on samples with similar doping concentrations can be found in \cite{dean67}. \cite{dean67} observe a line around 1.0924eV which is attributed to TO assisted boron bound exciton recombination (B$^{TO}$). Concentration values for B in \cite{dean67} are 6\cdot10$^{16}$~cm$^{-3}$. Also observed is a phosphorus line at 1.0916~eV, with 8\cdot10$^{16}$cm$^-3$ phosphorus atoms. With ES1 and MH2 having similar B and P values, they are expected to show a similar behavior. (See figure \ref{fig:boronSiPL} and \ref{fig:PSiPL} in appendix \ref{appendix:tabeller})

There is a photoluminescence line involving carbon bound to oxygen in Czochralski silicon known as the C-O band \cite{davies88}. In \cite{hare72}, it was observed only in crucible grown silicon, but not in float zone. In the crucible grown silicon, the oxygen impurities where 2\cdot10$^{18}$~atoms/cm$^3$, which is over ten times more than in ES1 and MH2. This makes it unlikely that any C-O complex luminescence will be strong enough to be detectable in these samples.

Another line involving carbon, is the two-carbon atom band \cite{davies88}. This band has been detected in float-zone silicon with C = 9.7\cdot10$^{16}$~cm$^{-3}$ after irradiation, together with the C-O complex line. The relative intensity between the C-O band and the two-carbon atom band in \cite{davies88} show that the C-C band at 969~meV is close to 5 times larger than the C-O band at 789~meV. With both MH2 and ES1 having carbon impurities around 6\cdot10$^{17}$~cm$^{-3}$, it is possible that this line at 969~meV will be visible. 

As for aluminum, \cite{dean67} show a line at 1.09~eV named Al$^{TO}$ in a sample with 2\cdot10$^{16}$cm$^{-3}$ Al doping atoms. In ES1 and MH2, the Al impurities are 20 times less. In addtion to a fairly low value of Al impurities, the Al$^{TO}$ line is very close to the intrinsic TO assisted band to band recombination I$^{TO}$, which can make it difficult to detect, and not likely to show up in the results.

Fe bound with boron is also known to give rise to photoluminescence at 1.069~eV \cite{mohring83}. The sample used in the article had  10$^{13}$ to 10$^{16}$~cm$^{-3}$ boron doping concentration. The article doesn't mention how many Fe impurity atoms that's introduced into the sample, but it's done by high temperature diffusion, and assumed to be considerably larger than for all the samples in this study. 

Based on the low values of Fe impurities in these samples, it's assumed that interstitial Fe won't have any effect on the photoluminescence bands. The same goes for Ti, which also have a very low amount present.


\subsubsection{Sample with added Chromium}

This sample have the same impurity values as ES1, except for chromium. The closest comparison is samples used in \cite{conzelmann82}. Here, luminescence spectra was observed for chromium in an p-type sample. Interstitial chromium concentrations where between $10^{14}$ and $10^{16} $~cm$^{-3}$ in \cite{conzelmann82}.

Chromium in an n-type sample doped with phosphorus atoms does not result in any luminescence, but chromium bound with boron show a clear line at 0.8432eV (CrB$^0$). The reaction velocity for the formation of CrB pairs at room temperature depend on the boron concentration. For large (10$^{15}$cm$^{-3}$) boron content, the chromium-boron reaction reach saturation in less than a day after chromium diffusion \cite{conzelmann82}.

There is enough boron atoms in MH2 to saturate chromium by forming CrB pairs. Chromium atoms are in the order of 10$^{14}$~atoms/cm$^3$ which is similar to that in \cite{conzelmann82}. Expected photoluminescence spectra is therefor expected to be similar. (See figure \ref{fig:CrBSiPL} in the appendix). Only a small amount of the overall boron atoms will be bound to chromium. There are also Fe impurities present in the sample, that can form bonds with boron. It has been shown experimentally that neutral iron does not form FeB pairs in boron-compensated n-type silicon \cite{lemke81}. Based on the low amount of Fe in this sample, lines related to Fe impurities are not believed to have any impact on the photoluminescence. 

