\section{Experimental}

Focus areas: Contaminants from Chromium, Titanium, Iron.


\subsection{Samples}

4 Samples: All polished and etched on top

Chromium contaminated, less than 10^13 cm^-3

Boron and Phosphorus equally distributed around 10^16 cm^-3.

Regular MC solar grade sample

Pure Si sample (FZ?)


\subsection{Pumping wavelength}

Pumping light needs to have enough energy to fill all available states in the crystal lattice, in order to detect defects and impurities. For silicon, which has a bandgap of around 1.1eV, has most impurity/defect bands below the bandgap. In order to fill these states, the pumping wavelength should be below 1125nm, which corresponds to energies just over 1.1eV.

Silicon has different absorption lengths for different wavelengths. For 1125nm, the absorption depth is nearly 200$�$m \cite{laserdybde}. Compared to absorption for 532nm, 1125nm reach 200 times deeper into the sample.

Absorption length of about 1�m for 532 nm laser, means that iron precipitates deeper in the sample won't be detected \cite{gundel09}. This limitation might be overcome by an excitation laser with a longer wavelength and absorption length in silicon. \cite{lee09} report that small angle grain boundaries in multicrystalline silicon of 1$^\circ$-1.5$^\circ$ show D3 and D4 lines, while 2$^\circ$-2.5$^\circ$ show D1 and D2 lines. Comparing to data from electron beam induced current measurements show D1 and D2 lines to be correlated with shallow levels, while D3 and D4 appear in both shallow and deep levels \cite{lee09}.

\subsection{Spot size}

At some distinct spots of a size between 1�m and 4�m the band to band photoluminescence peak is particular low at spots with iron precipitates  \cite{gundel09}. \cite{satoshi04} show that electron hole droplets become more intense for a smaller volume, with a silicon nanolayer smaller than the absorption depth of the laser. A 488nm pumping laser with 1,5�m diameter was used on different silicon nanolayer thickness in \cite{satoshi04}. That means the the total volume absorbing the laser is 0.75$�$m$^2$ (absorption depth of 0.5$�$m \cite{laserdybde}). With a larger volume, it would be a larger number of states for the laser to fill. In order to fill all the availible states in a larger volume, the laser intensity needs to be increased accordingly.

\subsection{Laser intensity}

With a large pumping intensity, an electron hole droplet become visible in the specter around 1.08eV in bulk silicon \cite{hammond75}. \cite{satoshi04} show that electron hole droplets occur at weak excitations (0.75mW) and even at high temperatures for a silicon nanolayer of 50nm. For thickness of 340nm, the electron hole droplet show up at pumping intensity of 3mW and above, and the intensity of the electron hole droplet grow larger than for the free exciton at 15mW.