\section{Experimental}

Focus areas: Contaminants from Chromium, Titanium, Iron.



4 Samples: All polished and etched on top

Chromium contaminated, less than 10^13 cm^-3

Boron and Phosphorus equally distributed around 10^16 cm^-3.

Regular MC solar grade sample

Pure Si sample (FZ?)


\subsection{Spot size}

At some distinct spots of a size between 1�m and 4�m the band to band photoluminescence peak is particular low at spots with iron precipitates  \cite{gundel09}. \cite{satoshi04} show that electron hole droplets become more intense for a smaller volume, with a silicon nanolayer smaller than the absorbtion depth of the laser. A 488nm pumping laser with 1,5�m diameter was used on different silicon nanolayer thickness.


\subsection{Pumping wavelength}

Absorption length of about 1�m for 532 nm laser, means that iron precipitates deeper in the sample won't be detected \cite{gundel09}. This limitation might be overcome by an excitation laser with a longer wavelength and absorption length in silicon.

\subsection{Laser intensity}

With a large pumping intensity, an electron hole droplet become visible in the specter around 1.08eV in bulk silicon \cite{hammond75}. \cite{satoshi04} show that electron hole droplets occur at weak excitations (0.75mW) and even at high temperatures for a silicon nanolayer of 50nm. For thickness of 340nm, the electron hole droplet show up at pumping intensity of 3mW and above, and the intensity of the electron hole droplet grow larger than for the free exciton at 15mW.