\section{Conclusion}

Literature review show a wide variety of silicon solar cell material properties that can be characterized using photoluminescence. A list of relevant spectra for the samples in question has been compiled, and can be found in the appendix, along with plots for relevant defects and impurities. Different production methods influence not only defects and concentrations of impurities, but also how these interact with each others. Impurities reacting with dislocations, and with other impurities produce a different luminescence spectra than interstitial impurities.


Noise is a considerable issue when analyzing the results. Dead pixels in the camera can be corrected, and even tho the dark current and background noise can be subtracted by measuring them, these are subject to change in between measurements, causing unwanted offset and peaks in the results. The peaks are disregarded, and identified by having the exact same shape and intensity for different excitation intensities. Offset from dark current is recognized as a signal at energies without any known luminescence spectra, like above the bandgap. The mean offset is of about the same value for all the intervals.

Exciting with intensities around 20~mW, and temperatures below 50K is required in order to unequivocally identify photoluminescence from impurities in the samples in this study. Samples with both P and B doping atoms can be identified by a luminescence peak at 1.04~eV when exciting the sample using low intensities, in addition to the bound exciton line at 1.092~eV, which has contributions from B and/or P. 


Dislocation luminescence known as D1/D2 are not visible in the samples. This is likely to be due to a low amount of metallic impurities at the 
dislocations \cite{arguirov07}. Lack of oxygen impurities would also influence defect luminescence in this region \cite{inoue07}. D3/D4 appear to be present in the samples on dislocation lines, and defect dots, giving rise to luminescence at 1.0~eV, and 0.95~eV, but are not clearly resolved.

The clean sample R6, show intrinsic properties comparable to those in \cite{dean67}, and luminescence attributed to dislocations is observed. In one of the grain boundaries of R6, a line is observed where luminescence from carbon-carbon complexes are known to appear. This can be due to the gathering of impurity atoms in the grain boundary, causing carbon complexes to form.

ES1 sample has considerably more radiative recombination than the other two samples. Expected behavior is that the clean sample R6, would have the most luminescence. The reason for ES1 to show increased photovoltaic properties is not known, and further studies are needed to determine the cause of this.

MH2 show signs of having a lower thermal conductivity than the other two samples based on increased broadening with respects to energy of the TO line. Chromium-boron pairs are unlikely to have been formed in MH2, due to the lack of related photoluminescence lines attributed to CrB pairs. This implies that chromium is mostly dissolved in the silicon lattice as an interstitial specie. The bound exciton line is considerably lower compared to TO in MH2, than for ES1, which suggest a higher concentration of P and B atoms in ES1. As this is not the case (showed by \cite{hystad09}), the difference can be attributed to a larger amount of dislocations in MH2, compared to ES1, in addition to increased local heating in MH2, compared to ES1.

Local heating appears to be a severe problem using micro photoluminescence. This heating is undesirable, making it harder to accurately characterize the sample. Using lower excitation intensity and longer integration times can overcome parts of the problem with local heating, but then the signal is likely to be close to the noise floor, making it hard to detect. In other words, micro photoluminescence has a disadvantage when it comes to local heating, compared to macro photoluminescence.