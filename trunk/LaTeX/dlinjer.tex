\subsection{Dislocation lines}

Several investigations have documented that dislocations in silicon give rise to characteristic photoluminescence (PL) spectra below the band edge. First showed in \cite{drozdov76} which labeled them D1 (0.812eV), D2 (0.875eV), D3 (0.934eV) and D4 (1.000eV). The samples where deformed at 850$^\circ$ C by bending, so that dislocation densities was inhomogeneous along the samples. \cite{drozdov76} states that the intensity of theese lines increase when the dislocation-rich parts of the crystal is approached. At the same time the intensity of the intrinsic characteristics decrease. The distance between D1-D4 (62 $\pm$ 3 meV) corresponds to the energy of the optical phonons in silicon \cite{drozdov76}. \cite{drozdov76} reports D1 and D2 are dominant in heavily deformed Si crystals, while D3 and D4 predominate in weakly deformed Si. A similar result was also reported by \cite{lee09} for small angle grain boundaries using cathodoluminescence.

\cite{sauer85} suggest that D1-D4 are due to dislocations which have been frozen in under low-shear stress. Photoluminescence under uniaxial stress shows that D1/D2 originate in the tetragonal defect with random orientation relative to <100> directions. \cite{sauer85} conclude that D3 and D4 are closely related, whereas the independent D1/D2 centers might be deformation-produced point defects in the strain region of dislocations. New lines D5 and D6 emerge when high-temperature, low-stress deformation is followed by low-temperature, high-stress deformation. \cite{sauer85} propose that line D5 is due to straight dislocations and D6 is due to stacking faults. \cite{sauer85} also suggest that D3/D4 photoluminescence is much more characteristic of the dislocations themselves than the D1/D2 emission lines. \cite{weronek91} state that D5 is correlated with electron-hole recombination at localized centers on separate partial dislocations. After annealing at moderate temperatures (T > 360$^\circ$C) the new lines merge into D4 \cite{weronek91}.


The origin of D1 and D2 is not clear. It has been argued that they originate in electronic transition at the geometrical kinks on dislocations \cite{suezawa83}, point defects \cite{sauer85} and impurities \cite{higgs91} and/or from the reaction products of dislocations \cite{sekiguchi95}. On the other hand, D3 and D4 lines is generally thought to be related to electronic transition within dislocation cores \cite{kveder95}. In addition, it has been suggested that the D3 line most likely is a phonon-assisted replica of D4 \cite{kveder95}.


Both \cite{drozdov76} and \cite{sauer85} studied plastically deformed silicon made by the Czochralski process (Cz-Si). \cite{tarasov00} studied  dislocations in multicrystalline silicon (mc-Si) and found similar lines with the entire set of D-lines shifted with around 10meV, presumably due to a strain field. Using a laser annealing technique, \cite{staiger94}, to introduce dislocations on a Cz-Si wafer, confirm the band location of D1-D4 from \cite{sauer85} in \cite{tarasov00}. A principal difference between dislocation D'-lines in mc-Si versus D-lines in Cz-Si is a substantial broadening (60-70meV at 77K) of the D1'/D2' lines \cite{tarasov00}.


\begin{table}[H]
\centering
\begin{tabular}{|c|c|c|c|c|}
\hline
Cz-Si \cite{drozdov76} & D1 & D2 & D3 & D4 \\
	& 0.812eV & 0.875eV & 0.934eV & 1.000eV \\
\hline
mc-Si \cite{tarasov00} & D1' & D2' & D3' & D4' \\
		& 0.80eV & 0.89eV & 0.95eV & 1.00eV \\
\hline
\end{tabular}
\caption{Energy positions of dislocation D-lines in Cz-Si and D' bands in mc-Si}
\label{tarasovlines}
\end{table}


\cite{tarasov00} reveal a linear dependence of the band-to-band photoluminescence intensity and minority carrier lifetime across entire multicrystalline-Si wafers. Photoluminescence mapping in \cite{tarasov00} of the 0.78eV (0.8eV) band intensity reveal a linkage to areas of a high dislocation density. This band should also be visible in room temperature \cite{tarasov00}.

\cite{tarasov01} later found that if the contamination level is too low, or too high (dislocation decorated by metal silicate precipitates) the defect photoluminescence band vanished in room temperature. However, a relatively low contamination level of dislocations, in the order of 10 impurity atoms per micron of the dislocation length produces distinguishable defect band luminescence \cite{tarasov01}. 

Dislocation related lines (D-lines) has been observed in low temperature photoluminiscence spectra from the regions which included the intragrain defects \cite{sugimoto06}. They also conclude that grain boundaries are not active recombination centers. \cite{sugimoto06} also show a TO-phonon replica of the boron bound exiton at 1.093eV. Intensity of boron bound exiton from the long lifetime regions was higher than that from the short lifetime regions. D-lines reported by \cite{sauer85} are in a short lifetime region. For a long lifetime region, \cite{sugimoto06} observe a peak at 1.00eV which is not the D4 line, but the zonecenter optical phonon sideband of the two-hole transition in the boron bound exiton \cite{dean67}. There have been no reports on the D-line spectrum missing only the D1 line \cite{sugimoto06}.


\cite{sugimoto07} study origins of the defects by low temperature photoluminescence spectroscopy, electron backscatter diffraction pattern measurement and the etch-pit observation, and conclude that defects are metal contaminated dislocation clusters which originated from small angle grain boundaries.





\subsection{Impurities}

Diffusion of transition metals into silicon crystals result in a variety of different electrically active levels in the forbidden bandgap.

% 'pure' impurities
\subsubsection{Atom impurities}

Copper doping of silicon crystals results in an intense emission at 1.014eV \cite{weber82}. \cite{weronek91} study Cu doped Si and observe a shoulder on the D1 line which presumably arises from Cu precipitates at the dislocation.

\cite{calao88} introduce Fe atoms into a float-zone silicon crystal and observe a spectrum of 0.735eV which relate to a complex defect containing iron.



Iron images in \cite{macdonald08} reveal internal gettering of iron to grain boundaries and dislocated regions during ingot growth.



% bounding stuff
\subsubsection{Impurities bound with doping atoms}

Silicon samples containing chromium-boron pairs exhibit characteristic luminescence lines in the 0.84eV region where the intensity increased linearly with laser power \cite{conzelmann83}. 

\cite{mohring83} observe a luminescence spectra around 1.07eV in boron-doped, iron-diffused crystalline silicon and suggest the source is B-Fe pairs.

% interaction with dislocations
\subsubsection{Interaction with dislocations}

Investigation in \cite{higgs92} show that transition-metal contamination plays an important role in the production of D-band luminescence from silicon samples containing either epitaxial stacking faults or oxidation-induced stacking faults. \cite{staiger94} found that Cu doping resulted in reduced intensity of D1 and D2, and the intensity of D3 and D4 become very small. \cite{weronek91} demonstrate that a complete passivation of the D-band luminescence is achieved at higher Cu and Fe concentration when deliberately contaminating high purity silicon samples which contain dislocations. However impurities like Ni, lead to no detectable changes in the spectrum \cite{weronek91}. D-band recombination in Si is found to be independent of impurities trapped at dislocations \cite{weronek91}, and \cite{sekiguchi95} concluded that metallic impurities don't seem to be related to D1 and D2 luminescence. % Even so, it is still generally accepted that metal impurity influence it. cite?

Room temperature mapping of the 0.77eV band is attributed to oxygen precipitates in in thermally treated silicon made by the Czochralski process (Cz-Si) \cite{tajima95}. This band peak shifts parallel to the bandgap with temperature. The increase of this band on the dislocation lines is due to the preferential precipitation of oxygen \cite{tajima95}.

\cite{inoue07} state that the deep-level emission from multicrystalline silicon with an intensity maximum at 0.78eV at room temperature is diffrent from that of the D1 line at low temperature. Furthermore, \cite{inoue07} suggest that the 0.78eV emission is associated with oxygen precipitation, and that the intra-grain defects are dislocation clusters decorated with oxygen impurities in addition to heavy-metal impurities.