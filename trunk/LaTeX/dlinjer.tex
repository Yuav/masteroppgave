Several investigations have documented that dislocations in silicon give rise to characteristic photoluminescence (PL) spectra below the band edge. First showed in \ref{drozdov76} which labeled them D1 (0.812eV), D2 (0.875eV), D3 (0.934eV) and D4 (1.000eV). The samples where deformed at 850 \degrees by bending, so that dislocation densities was inhomogeneous along the samples. \ref{drozdov76} states that the intensity of theese lines increase when the dislocation-rich parts of the crystal is approached. At the same time the intensity of the intrinsic characteristics decrease. The distance between D1-D4 (62 $\pm$ 3 meV) corresponds to the energy of the optical phonons in silicon \ref{drozdov76}.

\ref{sauer85} suggest that D1-D4 are due to dislocations which have been fronzen in under low-shear stress. ?? confirmed by M. Bugajski 1991??? Photoluminescence under uniaxial stress shows that D1/D2 originate in the tetragonal defect with random orientation relative to <100> directions. \ref{sauer85} conclude that D3 and D4 are closely related, whereas the independent D1/D2 centers might be deformation-produced point defects in the strain region of dislocations. New lines D5 and D6 emerge when high-temperature, low-stress deformation is followed by low-temperature, high-strss deformation. \ref{sauer85} propose that line D5 is due to straight dislocations and D6 is due to stacking faults. \ref{sauer85} also suggest that D3/D4 photoluminescence is much more characteristic of the dislocations themselves than the D1/D2 emission lines.

Both \ref{drozdov76} and \ref{sauer85} studied plastically deformed silicon made by the Czochralski process (Cz-Si). \ref{tarasov00} studied  dislocations in multicrystalline silicon (mc-Si) and found similar lines with the entire set of D-lines shifted with around 10meV, presumably due to a strain field. Using a laser annealing technique \ref{staiger94} to introduce dislocations on a Cz-Si wafer, confirm the band location of D1-D4 from \ref{sauer85} in \ref{tarasov00}. A principal difference between dislocation D'-lines in mc-Si versus D-lines in Cz-Si is a substantial broadening (60-70meV at 77K) of the D1'/D2' lines \ref{tarasov00}.


\begin{table}[H]
\centering
\begin{tabular}{|c|c|c|c|c|}
\hline
Cz-Si \ref{drozdov76} & D1 & D2 & D3 & D4 \\
	& 0.812eV & 0.875eV & 0.934eV & 1.000eV \\
\hline
mc-Si \ref{tarasov00} & D1' & D2' & D3' & D4' \\
		& 0.80eV & 0.89eV & 0.95eV & 1.00eV \\
\hline
\end{tabular}
\caption{Energy positions of dislocation D-lines in Cz-Si and D' bands in mc-Si}
\label{}
\end{table}


\ref{tarasov00} reveal a linear dependence of the band-to-band photoluminescence intensity and minority carrier lifetime across entire multicrystalline-Si wafers. Photoluminescence mapping in \ref{tarasov00} of the 0.78eV (0.8eV) band intensity reveal a linkage to areas of a high dislocation density. This band should also be visible in room temperature \ref{tarasov00}.


Dislocation related lines (D-lines) has been observed in low temperature photoluminiscence spectra from the regions which included the intragrain defects \ref{sugimoto06}. They also conclude that grain boundaries are not active recombination centers. \ref{sugimoto06} also show a TO-phonon replica of the boron bound exiton at 1.093eV. Intensity of boron bound exiton from the long lifetime regions was higher than that from the short lifetime regions. D-lines reported by \ref{sauer85} are in a short lifetime region. For a long lifetime region, \ref{sugimoto06} observe a peak at 1.00eV which is not the D4 line, but the zonecenter optical phonon sideband of the two-hole transition in the boron bound exiton \ref{dean67}. There have been no reports on the D-line spectrum missing only the D1 line \ref{sugimoto06}.

\ref{sugimoto07} <-- metal impurities

\ref{inoue07} state that the deep-level emission with an intensity maximum at 0.78eV at room temperature is diffrent from that of the D1 line at low temperature. Furthermore, \ref{inoue07} suggest that the 0.78eV emission is associated with oxygen precipitation, and that the intra-grain defects are dislocation clusters decorated with oxygen impurities in addition to heavy-metal impurities.