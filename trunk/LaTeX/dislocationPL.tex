\subsection{Literature review of relevant spectra}

\subsubsection{Dislocation photoluminescence}

Several investigations have documented that dislocations in silicon give rise to characteristic photoluminescence (PL) spectra below the band edge. First showed for low temperatures in \cite{drozdov76}, which labeled them D1 (0.812eV), D2 (0.875eV), D3 (0.934eV) and D4 (1.000eV). The samples where deformed at 850$^\circ$ C by bending, so that dislocation densities was inhomogeneous along the samples. \cite{drozdov76} states that the intensity of these lines increase closer to the dislocation-rich parts of the crystal. At the same time the intensity of the intrinsic characteristics decrease. The distance between D1-D4 (62 $\pm$ 3 meV) corresponds to the energy of the optical phonons in silicon \cite{drozdov76}. \cite{drozdov76} reports that D1 and D2 are dominant in heavily deformed Si crystals, while D3 and D4 predominate in weakly deformed Si. A similar result was also reported by recent study \cite{lee09} for small angle grain boundaries using cathodoluminescence.

It has been suggested in \cite{sauer85} that D1-D4 are due to dislocations which have been frozen in under low-shear stress. \cite{sauer85} state that photoluminescence under uniaxial stress shows that D1/D2 originate in the tetragonal defect with random orientation relative to <100> directions. \cite{sauer85} conclude that D3 and D4 are closely related, whereas the independent D1/D2 centers might be deformation-produced point defects in the strain region of dislocations. D1 and D2 is known to be closely related, as well as D3 and D4, and there have been no reports on the D-line spectrum missing only the D1 line \cite{sugimoto06}.

The origin of D1 and D2 is not clear. It has been argued that they originate in electronic transition at the geometrical kinks on dislocations \cite{suezawa83}, point defects \cite{sauer85} and impurities \cite{higgs91} and/or from the reaction products of dislocations \cite{sekiguchi95}. On the other hand, D3 and D4 lines is generally thought to be related to electronic transition within dislocation cores \cite{kveder95}. In addition, it has been suggested that the D3 line most likely is a phonon-assisted replica of D4 \cite{kveder95}.

New lines D5 and D6 emerge when high-temperature, low-stress deformation is followed by low-temperature, high-stress deformation. \cite{sauer85} propose that line D5 is due to straight dislocations and D6 is due to stacking faults. \cite{sauer85} also suggest that D3/D4 photoluminescence is much more characteristic of the dislocations themselves than the D1/D2 emission lines. \cite{weronek91} state that D5 is correlated with electron-hole recombination at localized centers on separate partial dislocations. After annealing at moderate temperatures (T > 360$^\circ$C) the new lines merge into D4 \cite{weronek91}.

Both \cite{drozdov76} and \cite{sauer85} studied plastically deformed silicon made by the Czochralski process (Cz-Si). \cite{tarasov00} studied  dislocations in multicrystalline silicon (mc-Si) and found similar lines with the entire set of D-lines shifted with around 10meV, presumably due to a strain field. Using a laser annealing technique \cite{staiger94}, introducing dislocations on a Cz-Si wafer, confirm the band location of D1-D4 from \cite{sauer85} in \cite{tarasov00}. A principal difference between dislocation D'-lines in mc-Si versus D-lines in Cz-Si is a substantial broadening in regards to energy (60-70meV at 77K) of the D1'/D2' lines observed in mc-Si \cite{tarasov00}.

\begin{table}[H]
\centering
\begin{tabular}{|c|c|c|c|c|}
\hline
Cz-Si \cite{drozdov76} & D1 & D2 & D3 & D4 \\
	& 0.812eV & 0.875eV & 0.934eV & 1.000eV \\
\hline
mc-Si \cite{tarasov00} & D1' & D2' & D3' & D4' \\
		& 0.80eV & 0.89eV & 0.95eV & 1.00eV \\
\hline
\end{tabular}
\caption{Energy positions of dislocation D-lines in Cz-Si and D' bands in mc-Si}
\label{tarasovlines}
\end{table}

Photoluminescence mapping in \cite{tarasov00} of the 0.78eV (0.8eV) band intensity reveal a linkage to areas of a high dislocation density. This band should also be visible in room temperature \cite{tarasov00}. \cite{tarasov00} also reveal a linear dependence of the band-to-band photoluminescence intensity and minority carrier lifetime across entire multicrystalline-Si wafers.

Dislocation related lines (D-lines) has been observed in low temperature photoluminescence spectra from the regions which included the intragrain defects. \cite{sugimoto06} concluded that grain boundaries are not active recombination centers. \cite{sugimoto06} also show a TO-phonon replica of the boron bound exiton at 1.093eV. Intensity of boron bound exciton from the long lifetime regions was higher than that from the short lifetime regions. D-lines reported by \cite{sauer85} are in a short lifetime region. For a long lifetime region, \cite{sugimoto06} observe a peak at 1.00eV which is not the D4 line, but the zone center optical phonon sideband of the two-hole transition in the boron bound exciton \cite{dean67}. \cite{kitler02} conclude that a relatively low contamination level of dislocations in the order of 10 impurity atoms/mm of the dislocation length produces D1 defect luminescence at room temperature and also degrades both the band-to-band luminescence and minority-carrier diffusion length.

It is believed that the intra-grain defects observed in the photoluminescence mapping are dislocations decorated with the heavy metals \cite{sugimoto06}. \cite{tarasov01} found that if the contamination level is too low, or too high (dislocation decorated by metal silicate precipitates) the defect photoluminescence band vanished in room temperature. However, a relatively low contamination level of dislocations, in the order of 10 impurity atoms per micron of the dislocation length produces distinguishable defect band luminescence \cite{tarasov01,kitler02}. 

\cite{sugimoto07} conclude that defects are metal contaminated dislocation clusters which originated from small angle grain boundaries. \cite{sugimoto07} study origins of the defects by low temperature photoluminescence spectroscopy, electron backscatter diffraction pattern measurement and the etch-pit observation. \cite{arguirov07} demonstrate three areas of a sample with only D3 and D4 present, and conclude that this is due low concentration of metallic impurities.
