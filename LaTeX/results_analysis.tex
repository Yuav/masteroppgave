\subsection{Results analysis}

Comparison plots have been slightly filtered using weak Savitzky-Golay filtering to reduce white noise for easier comparison. The line map and bar plots are without this filtering.

\subsubsection{Comparing different locations on sample R6-Q3-210}


\begin{figure}[H]
\centering
\includegraphics[width=0.7\columnwidth]{R6_comparisons}
\caption[R6-Q3-210 comparisons]{Comparison of different locations on sample R6-Q3-210 A from results in figure \ref{fig:R6-Area3_clean_area-20s}, \ref{fig:R6-Area1_dislocation_line-20s}, and \ref{fig:R6-Area2_dislocation_dot-20s}.}
\label{fig:R6_comparisons}%
\end{figure}

\begin{figure}[H]
\centering
\includegraphics[width=0.7\columnwidth]{R6_comparisons_Darea}
\caption[R6-Q3-210 comparisons close]{A closer look on differences in R6-Q3-210 A from graph in figure \ref{fig:R6_comparisons} }
\label{fig:R6_comparisons_Darea}%
\end{figure}

\begin{figure}[H]
\centering
\includegraphics[width=0.7\columnwidth]{R6_comparisons_TO}
\caption[R6-Q3-210 comparisons close]{A closer look on differences in R6-Q3-210 A from graph in figure \ref{fig:R6_comparisons} }
\label{fig:R6_comparisons_TO}%
\end{figure}

\begin{figure}[H]
\centering
\includegraphics[width=0.7\columnwidth]{R6_comparisons_I0}
\caption[R6-Q3-210 comparisons close]{A closer look on differences in R6-Q3-210 A from graph in figure \ref{fig:R6_comparisons} }
\label{fig:R6_comparisons_I0}%
\end{figure}


The strong TO line and its replicas, has relative intensities similar to those found in \cite{dean67}. Low impurity levels  in the sample allows for comparisons with intrinsic characteristics from \cite{dean67} (figure \ref{fig:SiPL}). Comparison show TO + 2 Zone center phonon at 0.968~eV, TO + Zone center phonon at 1.0315~eV, intervalley phonon replicas around 1.07~eV, TO at 1.097~eV and TA at 1.365~eV. It is also possible that the peak at 1.54~eV is the ideally forbidden no phonon peak, but it is very close to the noise floor, and cannot be unequivocally identified. The intensity differences can be due to small differences in focus of the excitation light, or other small alignment changes when physically moving the sample.


TO + Zone center phonon line have 7~\% relative intensity to TO in \cite{dean67}, when pumped with 7~W from a mercury arc on to an etched surface of intrinsic silicon. The relative intensity of TO + Zone center phonon line is \~6~\% for both the dislocation line, and dislocation dot. The difference here could be due to a bound exciton line at 1.092~eV also influencing the TO line. This peak at 1.092~eV is attributed to the TO phonon assisted Si:B bound excitons \cite{sauer73}. The Si:B free exciton would be at 1.150~eV, with 10\% intensity relative to the TO. This line could be present, by judgning the R6 results, however due to the noise level, it is hard to know for sure, but the relative intensities would be in the same order of magnitude as in \cite{sauer73} if it is in fact the free exciton line.


\cite{tajima78} describe a relation between B$^{TO}$(BE)/I$^TO$(FE). The ratio in R6 is close to 0.1, which in \cite{tajima78} correspond to N$_A$-N$_D\approx$5$\cdot$10$^{11}$~cm$^{-3}$. R6 is in the order of 10$^{16}$~cm$^{-3}$, and with excitation intensity (150~mW in \cite{tajima78}), and temperature (liquid helium temperatures in \cite{tajima78}) being similar, there is clearly some mismatch between the samples. Other differences is \cite{tajima78} using an Ar ion laser, and having as spot size of 1mm. This should, however, not affect the ratio in between the two lines, unless the smaller volume give rise to other types of recombination, like electron-hole drops, or non radiative recombination. One possible explanation is that \cite{tajima78} where looking at compensated high purity CZ-Si and FZ-Si, and that the relation is invalid for mc-Si. Specifically \cite{drozdov76} report a decrease in bound exciton intensities when nearing dislocation areas, which could account for the mismatch to results in \cite{tajima78}.


There are some lines present, that doesn't relate to intrinsic silicon, like the line at 1.0~eV. In addition, there is an increased intensity of TO + 2 zone center phonons which is expected to be ~1~\% of the TO line \cite{dean67}. But this appear to be \~7~\% in the R6 dislocation line , and \~3~\% in the R6 dislocation dot. These are likely to originate from dislocations. D3 and D4 are known to appear at 0.934~eV and 1.00~eV respectively at 4.2~K in CZ-Si \cite{drozdov76}. D4 has not been observed without D3, and vice versa, and D3 is considered to be a phonon replica of D4 \cite{kveder95}. In addition, \cite{tajima95} report a broad background emission reaching from 0.9-1.0~eV with D3 and D4 present. In the R6 results, on the dislocation dot, there's signs of D3' and D4' lines at 0.95~eV, and 1.0~eV respectively. This corresponds to measurements on mc-Si with a temperature of 77K, from \cite{tarasov00}, which reports a shift in the energies when comparing CZ-Si and mc-Si. The D3 line is larger than D4, which can be explained by the small energy differences between the three, momentum conserving TO phonons, which cause three replicas slightly different in the their energy \cite{arguirov07}.

If, in fact, it is the D3' and D4' lines in the result, D1 and D2, which also are attributed to dislocation areas are completely missing. \cite{kitler02} state that a relatively low contamination level of dislocations in the order of 10 impurity atoms/mm of the dislocation length produces D1 defect luminescence at room temperature. \cite{arguirov07} state that D1 and D2 are most probably caused by the interaction of the dislocations with background metallic impurities. This is a probable reason, due to the low level of metallic impurities in R6.

When pumping at lower intensities, the relative intensity of D3' and D4' compared to TO is higher. This is most likely caused by the D3' and D4' lines reaching saturation for high intensities, effectively halting the rise of these bands, for higher pumping light intensities. These lines are also considerably smaller for a defect free area. The peaks from this area are likely to arise from defects below the surface. 

There are two peaks observed at 1.04~eV and 0.98~eV. The 0.98~eV line, is only distinguishable in figure \ref{fig:R6-A-Area4-intensities-grain_boundary-30s}, but is likely to be present in the other results as well, based on the shape of the dislocation related luminescence, which has a contribution to the photoluminescence in between D3' and D4' that can not be explained by phonon replicas of the TO line alone. These lines are known as R1BB and R2BB \cite{arguirov02,arguirov03}. \cite{arguirov03} (figure \ref{fig:R1BB} in appendix) observed these lines in FZ-Si as well as mc-Si, and suggest that they are most probably phonon replicas of the band edge emission with 1, and 2 phonons respectively.

The two peaks observed above 1.16~eV are of unknown origin. The background noise, and dark current has been subtracted, however, the fact that these peaks have the same intensity and shape for different pumping intensities, and have energies above the silicon bandgap, suggest that these are in fact not photoluminescence signal, but noise added by the system.



\subsubsection{Comparing different locations on sample ES1-Q3-201}

\begin{figure}[H]
\centering
\includegraphics[width=0.7\columnwidth]{ES1_comparisons}
\caption[ES1-Q3-201 comparisons]{Comparison of different locations on sample ES1-Q3-201 C from results in figure \ref{fig:ES1-Area1_dislocation_free-10s}, \ref{fig:ES1-Area4_dislocation_line-10s}, \ref{fig:ES1-Area2_dislocation_spot-10s}, and \ref{fig:ES1-Area3_grain_boundary-10s}. }
\label{fig:ES1_comparisons}%
\end{figure}

\begin{figure}[H]
\centering
\includegraphics[width=0.7\columnwidth]{ES1_comparisons_Darea}
\caption[ES1-Q3-201 comparisons close]{A closer look on differences in ES1-Q3-201 C from graph in figure \ref{fig:ES1_comparisons} }
\label{fig:ES1_comparisons_Darea}%
\end{figure}

\begin{figure}[H]
\centering
\includegraphics[width=0.7\columnwidth]{ES1_comparisons_TO}
\caption[ES1-Q3-201 comparisons close]{A closer look on differences in ES1-Q3-201 C from graph in figure \ref{fig:ES1_comparisons} }
\label{fig:ES1_comparisons_TO}%
\end{figure}

\begin{figure}[H]
\centering
\includegraphics[width=0.7\columnwidth]{ES1_comparisons_I0}
\caption[ES1-Q3-201 comparisons close]{A closer look on differences in ES1-Q3-201 C from graph in figure \ref{fig:ES1_comparisons} }
\label{fig:ES1_comparisons_I0}%
\end{figure}


Intrinsic values are present, and recognized as I$^{TO+20^\Gamma}$[0.968~eV], I$^{TO+0^\Gamma}$[1.0315~eV], I$^{TO}$[1.097~eV], and I$^{TA}$[1.1365~eV]. In addition there are some expected, less defined luminescence at the TO intervalley phonon replica energies of 1.151~eV and 1.074~eV, and intervalley I$^{TO+0^\Gamma}$ replica energy at 1.013~eV. No phonon luminescence at 1.1545~eV seems to disappear in the area of grain boundary and dislocation line. A possible explanation for this, is that the emitted photons carry enough energy to excite a new electron-hole pair. Due to the geometrical properties of dislocation etched line and grain boundary, there is an etch pit with less atoms blocking the excitation laser, compared to the case of a perfect crystal structure. This would result in the laser penetrating deeper into the sample, and the photons are emitted from a deeper region of the sample and are less likely to be emitted with a direct path out of the sample.

%%%%%%%%%%%%%%%%%%%%%%%%%%%%% why??
%Dislocation related luminescence does not show up as individual luminescence. 

Phosphorus no phonon line is observed at 1.1496~eV in \cite{dean67}. This line should be accompanied by a TO phonon assisted line at 1.0916~eV with three times the intensity. These are not resolved in the ES1 results, but given the strong bound exciton peak, it's likely that it is present. Boron has a similar characteristic, with the no phonon line at 1.1503~eV, and TO phonon assisted line at 1.0924~eV. The no phonon line should have around 0.5\% of the B$^{TO}$ line, which is a considerably larger ratio than that of the P$^{TO}$ line \cite{dean67}. Boron and phosphorus lines are indistinguishable from each other, but both are likely to influence the luminescence. 


As expected, there are no C-O line present. As for Fe bound with B, a sharp line a 1.068~eV should be present \cite{mohring83}. Even though it could appear like this line is present, it is more likely that this is an artifact from noise, as the same sharp peak shape can be observed in different parts of the spectra known to contain nothing but noise. Based on the repeating on this peak, it is possible that the mean dark current noise has shifted, in regards to intensity for some pixels, compared to the measured mean dark current which is subtracted. An example which shows that the mean dark current noise has shifted, is the repeating artifacts from 0.7~eV to 0.9~eV. As the spectra consist of several glued together spectra, the artifact is repeating itself. These spectra has been glued together with some overlap, so that one side of the spectra has been removed. The rising intensities at 1.3~eV for the red line, show the non-overlapping part of the spectra. For lower energies, these pixels have been disregarded, and replaced by overlap except for the last interval.



Pumping at lower intensities, a new band is observed. This band is likely to be present at higher pumping intensities as well, but saturates, and becomes hard to detect. The peak of this band is at 1.04~eV




\subsubsection{Comparing different locations on sample MH2-Q3-210}

\begin{figure}[H]
\centering
\includegraphics[width=0.7\columnwidth]{MH2_comparisons}
\caption[MH2-Q3-210 comparisons]{Comparison of different locations on sample MH2-Q3-210 B2 from results in figure \ref{fig:MH2-Area1-dislocation_free-20s}, \ref{fig:MH2-Area2-dislocation_line-20s}, and \ref{fig:MH2-Area3-dislocation_dot-20s} }
\label{fig:MH2_comparisons}%
\end{figure}


\begin{figure}[H]
\centering
\includegraphics[width=0.7\columnwidth]{MH2_comparisons_Darea}
\caption[MH2-Q3-210 comparisons close]{A closer look on differences in MH2-Q3-210 B2 from graph in figure \ref{fig:MH2_comparisons} }
\label{fig:MH2_comparisons_Darea}%
\end{figure}

\begin{figure}[H]
\centering
\includegraphics[width=0.7\columnwidth]{MH2_comparisons_TO}
\caption[MH2-Q3-210 comparisons close]{A closer look on differences in MH2-Q3-210 B2 from graph in figure \ref{fig:MH2_comparisons} } 
\label{fig:MH2_comparisons_TO}%
\end{figure}

\begin{figure}[H]
\centering
\includegraphics[width=0.7\columnwidth]{MH2_comparisons_I0}
\caption[MH2-Q3-210 comparisons close]{A closer look on differences in MH2-Q3-210 B2 from graph in figure \ref{fig:MH2_comparisons} } 
\label{fig:MH2_comparisons_I0}%
\end{figure}


The no phonon line at 1.545~eV is not observed. TA, TO, BE, intervally phonon replica, and zone center phonon TO replica lines are recognized from figure \ref{fig:SiPL}. In addition, there are some luminescence around 1~eV and below, which cannot be attributed to 2 zone center phonon TO replica alone. If this is due to D4, which can be observed at 1.0~eV, then D3 should also be observed at 0.934~eV. While it is still possible these lines contribute to the intensity, the luminescence cannot be explained by these alone, due to the asymmetrical shape. \cite{arguirov07} observe a similar behavior around D4 at 80K, and argue that it indicates residual stress leading to strong multi-phonon (two and three) mediated band-to-band luminescence rather than radiation from defects.


CrB$^0$ from \cite{conzelmann82} where expected at 0.8432~eV. This is not the case for MH2. Chromium not bound with boron, does not to give rise to any luminescence, and \cite{conzelmann82} did not find any chromium related signal in phosphorous doped samples. \cite{conzelmann82} state that the intensity of this line is proportional with the chromium boron pairs. This would mean that the formation of these pairs haven't taken place in MH2. \cite{conzelmann82} used chromium doped (by diffusion) Si:B FZ and CZ samples stored at room temperature, to give CrB pairs time to bond. MH2 is mc-Si, and CrB pairs in MH2 have been investigated in \cite{hystad09} by measuring lifetime. No change in lifetime occurred by regards to time, which also suggest that the forming of these pairs have not taken place in MH2 to the extent that it is detectable. This implies that chromium is mostly dissolved in the silicon lattice as an interstitial specie.








%% Line plots

\begin{figure}[H]
\centering
\includegraphics[width=0.7\columnwidth]{MH2-mapping-bars-small_steps-20s}
\caption[MH2-Q3-210 line mapping]{Sample MH2-Q3-210 B2 pumped with 170mW at 14K line map using 10 small steps, looking at TO and BE line only from results in figure \ref{fig:MH2-mapping-small_steps-20s}.}
\label{fig:MH2-mapping-bars-small_steps-20s}%
\end{figure}

\begin{figure}[H]
\centering
\includegraphics[width=0.7\columnwidth]{MH2-mapping-bars-5_small_steps-20s}
\caption[MH2-Q3-210 line mapping]{Sample MH2-Q3-210 B2 pumped with 170mW at 14K line map using 20 small steps exactly 5 times larger than in figure \ref{fig:MH2-mapping-small_steps-20s}, looking at TO and BE line only from results in figure \ref{fig:MH2-mapping-5_small_steps-20s}.}
\label{fig:MH2-mapping-bars-5_small_steps-20s}%
\end{figure}


\begin{figure}[H]
\centering
\includegraphics[width=0.7\columnwidth]{MH2-mapping-bars-TOonBE-5_small_steps-20s}
\caption[MH2-Q3-210 line mapping]{Sample MH2-Q3-210 B2 comparing relative intensity between TO and BE line from results in \ref{fig:MH2-mapping-bars-5_small_steps-20s}.}
\label{fig:MH2-mapping-bars-TOonBE-5_small_steps-20s}%
\end{figure}


\subsubsection{Comparing similar areas on different samples}


\begin{figure}[H]
\centering
\includegraphics[width=0.7\columnwidth]{clean_area-20s}
\caption[Comparisons in a clean area]{Results from figure ( \ref{fig:R6-Area3_clean_area-20s},\ref{fig:ES1-Area1_dislocation_free-10s},\ref{fig:MH2-Area1-dislocation_free-20s}) where results in figure \ref{fig:ES1-Area1_dislocation_free-10s} are multiplied by 2, to account for 10s integration time compared to 20.}
\label{fig:clean_area-20s_comparison}%
\end{figure}

\begin{figure}[H]
\centering
\includegraphics[width=0.7\columnwidth]{clean_area-20s_zoom}
\caption[Comparisons in a clean area]{Plot from figure \ref{fig:clean_area-20s_comparison} in greater detail}
\label{fig:clean_area-20s_zoom_comparison}%
\end{figure}

\begin{figure}[H]
\centering
\includegraphics[width=0.7\columnwidth]{dislocation_dot-20s}
\caption[Comparisons in a dislocation dot]{Results from figure ( \ref{fig:R6-Area2_dislocation_dot-20s},\ref{fig:ES1-Area2_dislocation_spot-10s},\ref{fig:MH2-Area3-dislocation_dot-20s}) where results in figure \ref{fig:ES1-Area2_dislocation_spot-10s} are multiplied by 2, to account for 10s integration time compared to 20. }
\label{fig:dislocation_dot-20s_comparison}%
\end{figure}

\begin{figure}[H]
\centering
\includegraphics[width=0.7\columnwidth]{dislocation_dot-20s_zoom}
\caption[Comparisons in a dislocation dot]{Plot from figure \ref{fig:dislocation_dot-20s_comparison} in greater detail}
\label{fig:dislocation_dot-20s_zoom_comparison}%
\end{figure}

\begin{figure}[H]
\centering
\includegraphics[width=0.7\columnwidth]{dislocation_line-20s}
\caption[Comparisons in a dislocation line]{Results from figure ( \ref{fig:R6-Area1_dislocation_line-20s},\ref{fig:ES1-Area4_dislocation_line-10s},\ref{fig:MH2-Area2-dislocation_line-20s}) where results in figure \ref{fig:ES1-Area4_dislocation_line-10s} are multiplied by 2, to account for 10s integration time compared to 20. }
\label{fig:dislocation_line-20s_comparison}%
\end{figure}


\begin{figure}[H]
\centering
\includegraphics[width=0.7\columnwidth]{dislocation_line-20s_zoom}
\caption[Comparisons in a dislocation line]{Plot from figure \ref{fig:dislocation_line-20s_comparison} in greater detail}
\label{fig:dislocation_line-20s_zoom_comparison}%
\end{figure}

\begin{figure}[H]
\centering
\includegraphics[width=\columnwidth]{TO_comparisons}
\caption[Comparison or relative strength]{Comparison of relative strength of known characteristics to TO line}
\label{fig:TO_comparisons}%
\end{figure}




Compared to R6, ES1 have a fairly intense bound exciton line at 1.092~eV. This corresponds to a higher concentration of boron and phosphorus atoms according to \cite{tajima78}. The ratio is different on different parts of the sample, which could mean that the doping atoms are unevenly distributed. It could also mean that the excitons at defect locations are more subject to interaction with defect characteristics, than the TO line. This behavior was reported in \cite{drozdov76}, which state that the series of bound exciton lines decrease sharply in intensity in areas of dislocations.