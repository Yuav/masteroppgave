\section{Results analysis}

\subsection{Comparing different locations on sample MH2-Q3-210}

\begin{figure}[H]
\centering
\includegraphics[width=0.7\columnwidth]{MH2-mapping-bars-small_steps-20s}
\caption[MH2-Q3-210 line mapping]{Sample MH2-Q3-210 B2 pumped with 170mW at 14K line map using 10 small steps, looking at TO and BE line only from results in figure \ref{fig:MH2-mapping-small_steps-20s}.}
\label{fig:MH2-mapping-bars-small_steps-20s}%
\end{figure}

\begin{figure}[H]
\centering
\includegraphics[width=0.7\columnwidth]{MH2-mapping-bars-5_small_steps-20s}
\caption[MH2-Q3-210 line mapping]{Sample MH2-Q3-210 B2 pumped with 170mW at 14K line map using 20 small steps exactly 5 times larger than in figure \ref{fig:MH2-mapping-small_steps-20s}, looking at TO and BE line only from results in figure \ref{fig:MH2-mapping-5_small_steps-20s}.}
\label{fig:MH2-mapping-bars-5_small_steps-20s}%
\end{figure}


\begin{figure}[H]
\centering
\includegraphics[width=0.7\columnwidth]{MH2-mapping-bars-TOonBE-5_small_steps-20s}
\caption[MH2-Q3-210 line mapping]{Sample MH2-Q3-210 B2 comparing relative intensity between TO and BE line from results in \ref{fig:MH2-mapping-bars-5_small_steps-20s}.}
\label{fig:MH2-mapping-bars-TOonBE-5_small_steps-20s}%
\end{figure}

\subsection{Comparing similar areas on different samples}


\begin{figure}[H]
\centering
\includegraphics[width=0.7\columnwidth]{clean_area-20s}
\caption[Comparisons in a clean area]{Results from figure ( \ref{fig:R6-Area3_clean_area-20s},\ref{fig:ES1-Area1_dislocation_free-10s},\ref{fig:MH2-Area1-dislocation_free-20s}) where results in figure \ref{fig:ES1-Area1_dislocation_free-10s} are multiplied by 2, to account for 10s integration time compared to 20. Results also have weak Savitzky-Golay filtering to reduce white noise for easier comparison.}
\label{fig:clean_area-20s_comparison}%
\end{figure}


\begin{figure}[H]
\centering
\includegraphics[width=0.7\columnwidth]{dislocation_dot-20s}
\caption[Comparisons in a dislocation dot]{Results from figure ( \ref{fig:R6-Area2_dislocation_dot-20s},\ref{fig:ES1-Area2_dislocation_spot-10s},\ref{fig:MH2-Area3-dislocation_dot-20s}) where results in figure \ref{fig:ES1-Area2_dislocation_spot-10s} are multiplied by 2, to account for 10s integration time compared to 20. Results also have weak Savitzky-Golay filtering to reduce white noise for easier comparison.}
\label{fig:dislocation_dot-20s_comparison}%
\end{figure}



\begin{figure}[H]
\centering
\includegraphics[width=0.7\columnwidth]{clean_area-20s}
\caption[Comparisons in a dislocation line]{Results from figure ( \ref{fig:R6-Area1_dislocation_line-20s},\ref{fig:ES1-Area4_dislocation_line-10s},\ref{fig:MH2-Area2-dislocation_line-20s}) where results in figure \ref{fig:ES1-Area4_dislocation_line-10s} are multiplied by 2, to account for 10s integration time compared to 20. Results also have weak Savitzky-Golay filtering to reduce white noise for easier comparison.}
\label{fig:dislocation_line-20s_comparison}%
\end{figure}
