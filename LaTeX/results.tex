\section{Results}

Results have been corrected for dead pixels, measured dark current, and measured background noise unless otherwise specified. Grating is fixed at 300, and pumping wavelength is 800nm. Counts is a number given from the spectrometer which relate to the relative intensity detected by the pixel in the CCD.

\subsection{R6-Q3-210}

This is the electronic grade sample, with a very low amount of impurities.

%% Light microscope pictures, with zoom on areas


\begin{figure}[H]
\centering
\includegraphics[width=\columnwidth]{R6-Area1_dislocation_line-20s}
\caption[R6-Q3-210 at a dislocation line]{Sample R6-Q3-210 A pumped with 170mW at 10K in a dislocation line (Area 1).}
\label{fig:R6-Area1_dislocation_line-20s}%
\end{figure}


\begin{figure}[H]
\centering
\includegraphics[width=\columnwidth]{R6-Area2_dislocation_dot-20s}
\caption[R6-Q3-210 at a dislocation dot]{Sample R6-Q3-210 A pumped with 170mW at 10K in a dislocation dot (Area 2).}
\label{fig:R6-Area2_dislocation_dot-20s}%
\end{figure}


\begin{figure}[H]
\centering
\includegraphics[width=\columnwidth]{R6-Area3_clean_area-20s}
\caption[R6-Q3-210 at a dislocation free area]{Sample R6-Q3-210 A pumped with 170mW at 10K in a dislocation free area (Area 3).}
\label{fig:R6-Area3_clean_area-20s}%
\end{figure}

\subsection{ES1-Q3-201}

This sample is from a dirty feedstock, with large amount of P and B.

\subsubsection{Room temperature}

\begin{figure}[H]
\centering
\includegraphics[width=\columnwidth]{ES1-room_temperature}
\caption[ES1-Q3-201 at room temperature]{Sample ES1-Q3-201 A pumped with 26mW at 295K in a dislocation line (black) and in a clean area (blue). An estimated dark current offset has been subtracted.}
\label{fig:ES1-room_temperature}%
\end{figure}


\subsubsection{Low temperature}

\begin{figure}[H]
\centering
\includegraphics[width=\columnwidth]{ES1-Area1_dislocation_free-10s}
\caption[ES1-Q3-201 at a dislocation free area]{Sample ES1-Q3-201 C pumped with 170mW at 12K in a dislocation free area (Area 1).}
\label{fig:ES1-Area1_dislocation_free-10s}%
\end{figure}

\begin{figure}[H]
\centering
\includegraphics[width=\columnwidth]{ES1-Area2_dislocation_spot-10s}
\caption[ES1-Q3-201 at a dislocation free area]{Sample ES1-Q3-201 C pumped with 170mW at 12K in a dislocation spot (Area 2).}
\label{fig:ES1-Area2_dislocation_spot-10s}%
\end{figure}


\begin{figure}[H]
\centering
\includegraphics[width=\columnwidth]{ES1-Area3_grain_boundary-10s}
\caption[ES1-Q3-201 at a grain boundary]{Sample ES1-Q3-201 C pumped with 170mW at 12K in a grain boundary (Area 3).}
\label{fig:ES1-Area3_grain_boundary-10s}%
\end{figure}

\begin{figure}[H]
\centering
\includegraphics[width=\columnwidth]{ES1-Area3_grain_boundary_60s}
\caption[ES1-Q3-201 at a grain boundary]{Sample ES1-Q3-201 C pumped with 170mW at 12K in a grain boundary (Area 3) with the result in figure \ref{fig:ES1-Area3_grain_boundary-10s} plotted as the black line.}
\label{fig:ES1-Area3_grain_boundary_60s}%
\end{figure}


\begin{figure}[H]
\centering
\includegraphics[width=\columnwidth]{ES1-Area4_dislocation_line-10s}
\caption[ES1-Q3-201 at a dislocation line]{Sample ES1-Q3-201 C pumped with 170mW at 14K in a dislocation line (Area 4).}
\label{fig:ES1-Area4_dislocation_line-10s}%
\end{figure}

\begin{figure}[H]
\centering
\includegraphics[width=\columnwidth]{ES1-Area4_dislocation_line-60s}
\caption[ES1-Q3-201 at a dislocation line]{Sample ES1-Q3-201 C pumped with 170mW at 14K in a dislocation line (Area 4) with the result in figure \ref{fig:ES1-Area4_dislocation_line-10s} plotted as the black line. For 60s integration, the main TO line around 1.1eV is saturating the camera.}
\label{fig:ES1-Area4_dislocation_line-60s}%
\end{figure}

\subsection{MH2-Q3-210}

This sample is the same as ES1-Q3-201 except for added chromium in this one.

\subsubsection{Room temperature}

\begin{figure}[H]
\centering
\includegraphics[width=\columnwidth]{MH2-room_temperature}
\caption[MH2-Q3-210 at room temperature]{Sample MH2-Q3-210 C pumped with 26mW at 295K in a dislocation free area (blue), and in an area with dislocations (black). An estimated dark current offset has been subtracted, and results are Savitzky-Golay filtered for easier comparison.}
\label{fig:MH2-room_temperature}%
\end{figure}

\subsubsection{At 70K}
%%%TODO double check that it is in fact 20s integration, and not 10s due to low signal / noise floor
\begin{figure}[H]
\centering
\includegraphics[width=\columnwidth]{MH2-70K}
\caption[MH2-Q3-210 at 70K]{Sample MH2-Q3-210 D pumped with 170mW at 70K in a dislocation free area (blue), and in an area with dislocations (black). An estimated dark current offset has been subtracted, and results are Savitzky-Golay filtered for easier comparison.}
\label{fig:MH2-70K}%
\end{figure}

\subsubsection{Low temperature}

\begin{figure}[H]
\centering
\includegraphics[width=\columnwidth]{MH2-Area1-dislocation_free-20s}
\caption[MH2-Q3-210 at area 1]{Sample MH2-Q3-210 B2 pumped with 170mW at 12K in a dislocation free area (Area 1).}
\label{fig:MH2-Area1-dislocation_free-20s}%
\end{figure}

\begin{figure}[H]
\centering
\includegraphics[width=\columnwidth]{MH2-Area2-dislocation_line-20s}
\caption[MH2-Q3-210 at area 2]{Sample MH2-Q3-210 B2 pumped with 170mW at 12K in a dislocation line (Area 2).}
\label{fig:MH2-Area2-dislocation_line-20s}%
\end{figure}

\begin{figure}[H]
\centering
\includegraphics[width=\columnwidth]{MH2-Area3-dislocation_dot-20s}
\caption[MH2-Q3-210 at area 3]{Sample MH2-Q3-210 B2 pumped with 170mW at 12K in a dislocation dot (Area 3).}
\label{fig:MH2-Area3-dislocation_dot-20s}%
\end{figure}

\begin{figure}[H]
\centering
\includegraphics[width=\columnwidth]{MH2-Area3-dislocation_dot-60s}
\caption[MH2-Q3-210 at area 3]{Sample MH2-Q3-210 B2 pumped with 170mW at 12K in a dislocation dot (Area 3) with 20s integration time (black) and 60s integration time (blue). The 60s integration time has an estimated dark current offset subtracted, in addition to measured dark current due to mismatched dark current measurement.}
\label{fig:MH2-Area3-dislocation_dot-60s}%
\end{figure}

\subsubsection{Line mapping}

These results are a line mapping of different spots on the sample.

\begin{figure}[H]
\centering
\includegraphics[width=\columnwidth]{MH2-mapping-small_steps-20s}
\caption[MH2-Q3-210 line mapping]{Sample MH2-Q3-210 B2 pumped with 170mW at 14K line map using 10 small steps.}
\label{fig:MH2-mapping-small_steps-20s}%
\end{figure}

\begin{figure}[H]
\centering
\includegraphics[width=\columnwidth]{MH2-mapping-5_small_steps-20s}
\caption[MH2-Q3-210 line mapping]{Sample MH2-Q3-210 B2 pumped with 170mW at 14K line map using 20 steps exactly 5 times larger than in figure \ref{fig:MH2-mapping-small_steps-20s}.}
\label{fig:MH2-mapping-5_small_steps-20s}%
\end{figure}
