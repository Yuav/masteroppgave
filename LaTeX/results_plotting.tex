\subsection{Results plotting}

In order to plot the results from the spectrometer, there are a few manipulations that's needed.

\subsubsection{Disregarding defect pixels}

By taking a spectra with the shutter closed, it is possible to measure the dark current coming from the camera. The dark current should be equally distributed across the pixels, based on the assumption that all pixels behave the same. For long integration time, this is not the case as seen in figure \ref{fig:dark_current_40s}.

\begin{figure}[H]
\centering
\includegraphics[width=0.8\columnwidth]{Dark_current-40s}
\caption[Defective pixels]{Dark current signal from the camera with defective pixels with shutter closed, and CCD at -75$^\circ$C using a random center wavelength}%
\label{fig:dark_current_40s}%
\end{figure}

To solve this problem, the four pixels standing out are disregarded, and the value of the neighbor pixel has been used instead. Matlab code for this is available in the appendix. 

\begin{figure}[H]
\centering
\includegraphics[width=0.8\columnwidth]{Dark_current-40s_corrected}
\caption[Defective pixels corrected]{Dark current signal from the camera with defective pixel correction in red using a random center wavelength}%
\label{fig:dark_current_40s-corrected}%
\end{figure}

The defective pixels are less apparent for shorter integration time, but still a problem:

\begin{figure}[H]
\centering
\subfigure[Dark current without correction]{
\includegraphics[width=.45\columnwidth]{Dark_current-10s}
\label{fig:dark_current_10s}
}
\subfigure[Dark current with correction (red)]{
\includegraphics[width=.45\columnwidth]{Dark_current-10s_corrected}
\label{fig:dark_current_10s-corrected}
}
\label{fig:dark_current_correction_10s-parentfig}
\caption[Dark current with 10s integration time]{Dark current with 10s integration time using a random center wavelength}
\end{figure}

Dead pixel correction is performed in all of the results.

\subsubsection{Noise reduction}

As seen in the previous section, there is a mean increased count, due to a dark current in the results present. Ideally, all pixels should behave exactly the same, and give rise to the same dark current offset. If all pixels behaved the same, and with the same variance in between measurements, it could simply be subtracted. This is not the case. The dark current is unevenly distributed over the pixel array, and needs to be measured, order to remove it. The mean dark current noise shape is pretty much the same from measurement to measurement using the same CCD temperature. There is white noise elements, in addition to the dark current offset. 

\begin{figure}[H]
\centering
\includegraphics[width=\columnwidth]{Dark_current_and_background_noise-20s}
\caption[Dark current and noise]{Dark current (blue) and dark current + background noise (red) with Savitzky-Golay filtered noise floor estimation (cyan)}%
\label{fig:dark_current_and_background_noise}%
\end{figure}

By subtracting the filtered dark current found in the dark current noise measurement, only background noise and white noise would be left. The matlab code used to do this can be found in the appendix.

\begin{figure}[H]
\centering
\includegraphics[width=\columnwidth]{Dark_current_removed-20s}
\caption[Dark current removed]{Dark current removed from background noise (blue), and Savitzky-Golay filtered signal (cyan)}%
\label{fig:dark_current_removed-20s}%
\end{figure}

It appears that the dark current noise is larger with the shutter open, compared to closed. But a more critical noise in the spectrum is a background noise signal around 1064~nm. The spectrometer has a wavelength range of 140~nm when using 300 as grating. It has proven difficult to align the system so that the entire array of pixels in the camera get an equally distributed light beam. And based on the noise level, the wavelength interval is chosen as 100~nm, in order to remove the left hand side, with increased dark current offset, from the spectra when gluing different intervals together. This also avoid the problem of not hitting the entire array evenly. A full spectra from 800-1650~nm was done to map the light coming from the room, known as background noise. The only background noise peak visible in this wavelength area, is shown in figure \ref{fig:dark_current_removed-20s}.

\subsubsection{Savitzky-Golay filtering}

Savitzky-Golay filter (also called digital smoothing polynomial filter, or least-squares smoothing filter), is a smoothing filter which essentially performs a local polynomial regression on a series of values equally spaced. This filter is chosen because of the large frequency span of the signal. Although Savitzky-Golay filters are more effective at preserving the pertinent high frequency components of the signal, they are less successful than standard averaging finite impulse response (FIR) filters at rejecting noise \cite{signal_processing}.
