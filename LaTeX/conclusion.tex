\section{Conclusion}

Literature review show a wide variety of silicon solar cell material properties that can be characterized using photoluminescence. A list of relevant spectra for the samples in question has been compiled, and can be found in the appendix, along with plots for relevant defects and impurities. Different production methods influence not only defects and concentrations of impurities, but also how these interact with each others. Impurities reacting with dislocations, and with other impurities produce a different luminescence spectra than interstitial impurities.

Dislocation luminescence known as D1/D2 are not visible in the samples. This is likely to be due to a low amount of metallic impurities at the dislocations \cite{arguirov07}. Lack of oxygen impurities would also influence defect luminescence in this region \cite{inoue07}. D3/D4 appear to be present in the samples on dislocation lines, and defect dots, giving rise to luminescence at 1.0~eV, and 0.95~eV, but are not clearly resolved.

Exciting with intensities around 20~mW, and temperatures below 50K is required in order to unequivocally identify photoluminescence from defects and impurities in the samples in this study. Samples with both P and B doping atoms can be identified by a luminescence peak at 1.04~eV when exciting the sample using low intensities, in addition to the bound exciton line at 1.092~eV, which has contributions from B and/or P. 

MH2 show signs of having a lower thermal conductivity than the other two samples based on increased broadening with respects to energy of the TO line. Chromium-boron pairs are unlikely to have been formed in MH2, due to the lack of related photoluminescence lines attributed to CrB pairs. Even so, the bound exciton line is considerably lower compared to TO, than for ES1, which suggest an uneven relation with regards to P and B atoms, or dislocations, between ES1 and MH2.

Local heating appears to be a severe problem using micro photoluminescence. This heating is undesirable, making it harder to accurately characterize the sample. Using lower excitation intensity and longer integration times can overcome parts of the problem with local heating, but then the signal is likely to be close to the noise floor, making it hard to detect. In other words, micro photoluminescence has a disadvantage when it comes to local heating, compared to macro photoluminescence.