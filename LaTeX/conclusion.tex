\section{Conclusion}

Exciting with intensities around 20~mW, and temperatures below 50K is required in order to unequivocally identify photoluminescence from defects and impurities on the samples in this study.
Samples with both P and B doping atoms can be identified by a luminescence peak at 1.04~eV when exciting the sample using low intensities, in addition to the bound exciton line at 1.092~eV, which has contributions from B and/or P. Chromium-boron pairs are unlikely to have been formed in MH2, due to the lack of related photoluminescence lines attributed to CrB pairs.